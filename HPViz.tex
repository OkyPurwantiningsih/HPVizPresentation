\documentclass[xcolor=table]{beamer}
\usepackage{subfig}

\makeatletter
\def\normaljustify{%
  \let\\\@centercr\rightskip\z@skip \leftskip\z@skip%
  \parfillskip=0pt plus 1fil}
\makeatother
\usetheme{Madrid}



\titlegraphic{
    \includegraphics[width=1.5cm,height=1.5cm,keepaspectratio]{images/it4bi_logo.jpg}~~%
    \includegraphics[width=1.5cm,height=1.5cm,keepaspectratio]{images/logocs.jpeg}~~%
    \includegraphics[width=1.5cm,height=1.5cm,keepaspectratio]{images/Logo_univ_paul_valery.jpg}~~%
    \includegraphics[width=2.5cm,height=2.5cm,keepaspectratio]{images/lirmm_logo2.jpg}%
}

\title[VA on Human Body Movement Data]{Visual Analytics on Human Body Movement Data }
\subtitle{Applied on Healthcare}
\author[Oky Purwantiningsih]{Oky Purwantiningsih\\Advisors: Arnaud Sallaberry, Jer\^{o}m\'{e} Az\'{e}}
\institute[IT4BI]{Decision Support and Business Intelligence\\IT4BI Master Thesis\\Prepared at LIRMM and Universit\'{e} Paul Val\'{e}ry Montpellier\\In collaboration with NaturalPad}
\date{\today}


\begin{document}
\setbeamertemplate{caption}{\raggedright\insertcaption\par}

\begin{frame}
\titlepage
\end{frame}

\begin{frame}
\label{contents}
\frametitle{Outline}
\tableofcontents
\end{frame}

%=================================
\section{Introduction}
%\subsection{Motivation}
\begin{frame}
\frametitle{Motivation}
\begin{columns}
\column{0.5\textwidth}
\centering
\begin{figure}
%\includegraphics[width=3.5cm,height=1.7cm]{images/kinect.jpg}
%\caption{Kinect}
\subfloat[Kinect]{{\includegraphics[width=3.5cm,height=1.5cm]{images/kinect.jpg} }}
\end{figure}
\begin{figure}
%\includegraphics[width=3.5cm,height=1.7cm]{images/wii_balance_board.jpg}
%\caption{Wii Balance Board}
\subfloat[Wii Balance Board]{{\includegraphics[width=3.5cm,height=1.5cm]{images/wii_balance_board.jpg} }}
\end{figure}

\column{0.5\textwidth}
\centering
\begin{figure}
%\includegraphics[width=3.5cm,height=1.7cm]{images/wii_remote.jpg}
%\caption{Wii Remote}
\subfloat[Wii Remote]{{\includegraphics[width=3.5cm,height=1.5cm]{images/wii_remote.jpg} }}
\end{figure}
\begin{figure}
%\includegraphics[width=3.5cm,height=1.7cm]{images/play_station_move.jpg}
%\caption{Play Station Move}
\subfloat[Play Station Move]{{\includegraphics[width=3.5cm,height=1.5cm]{images/play_station_move.jpg} }}
\end{figure}

\end{columns}
\vspace{0.5cm}
Motion Sensing input devices enable players to control and interact with the game console through body movement.
\end{frame}

\begin{frame}
\frametitle{Motivation}
\centering

\begin{itemize}
\onslide<1->\item Hammer and Planks is a serious game developed by NaturalPad, designed to train the equilibrium of patient with balance disorders.\\
\onslide<2->\item Players are required to move their body parts to the right, left, backward, and forward as part of the rehabilitation\\
\begin{figure}\onslide<3->
\includegraphics[scale=0.16]{images/hp_game.png}
\end{figure}
\normaljustify
\onslide<4->\item Game missions: collect bonuses, kill enemies, avoid obstacles
\end{itemize}

\end{frame}

\begin{frame}
\frametitle{Motivation}
\begin{minipage}[t][0.5\textheight][t]{\textwidth}
\begin{figure}
\includegraphics[scale=0.35]{images/current_viz.png}
\caption{Current Visualization}
\end{figure}

\end{minipage}

\begin{minipage}[t][0.5\textheight][t]{\textwidth}
\only<1,2>{Available:
\begin{itemize}
\onslide<1->\item Average degree of body movement over x axis
\onslide<2->\item Average degree of body movement over y axis
\end{itemize}}

\only<3,4,5>{Not available:
\begin{itemize}
\onslide<3->\item How often the player move to right or left?
\onslide<4->\item To which type of events (collecting bonuses, killing enemies) the movement is related?
\onslide<5->\item Evolution of player's movement
\end{itemize}}
\end{minipage}
\end{frame}

%\subsection{Methodology}
\begin{frame}
\frametitle{Methodology}
\centering
\begin{figure}
\includegraphics[scale=0.7]{images/munzner_model.png}
\caption{Munzner's Visualization Design Model (Munzer, 2009)}
\end{figure}

\normaljustify
\begin{itemize}
\onslide<1->\item What kind of information needed by health professionals from the visualization (List of Tasks)
\onslide<2->\item Define data structure to support the tasks
\onslide<3->\item Define visualization and interaction technique
\onslide<4->\item Define algorithm to support the visualization
\end{itemize}
\end{frame}

%=================================
\section{Domain Problem Characterization}
%\subsection{Hammer and Planks Game Dynamic}
\begin{frame}
\frametitle{Hammer and Planks Game Dynamic}
\begin{columns}[T]
\column{0.6\textwidth}
\begin{itemize}
\onslide<1->\item Game difficulty can be changed by setting:
\begin{itemize}
\item \textbf{Number of objects}
\item \textbf{Area where objects can appear}
\item \textbf{Activity duration and repetition}
\item \textbf{Movement Type}
\item \textbf{Movement Direction}
\end{itemize}
\onslide<2->\item Movement Type:
\begin{itemize}
\item \textbf{BodyTilt}
\item \textbf{HandPoint}
\item \textbf{ShoulderCGE}
\end{itemize}
\onslide<3->\item Movement Direction:
\begin{itemize}
\item \textbf{Horizontal}: moves boat horizontally
\item \textbf{Vertical}: changes boat speed
\end{itemize}
\end{itemize}

\column{0.4\textwidth}
\normaljustify
\onslide<4->\begin{figure}
\includegraphics[scale=0.1]{images/mia_bodytilt.png}
\caption{BodyTilt}
\end{figure}

\end{columns}
\end{frame}

%\subsection{Target User Question}
\begin{frame}
\frametitle{Target User Question}
\onslide<1->(Q1) For a given session, to \textbf{which direction (right/left)} the player moved more?\\ 
\onslide<2->(Q2) For a given session, how does the player perform based on the \textbf{number of objects collected, avoided, or killed} with respect to the \textbf{area of the movement}?\\
\onslide<3->(Q3) For a given session, how does the player perform based on the \textbf{number of objects collected, avoided, or killed} with respect to the \textbf{area of movement and the speed} in which the game is played?\\ 
\onslide<4->(Q4) For a given patient, has he/she \textbf{improved in the game overtime}?\\ 
\onslide<5->(Q5) For a given patient, has he/she \textbf{improved in a certain area overtime}?


\end{frame}
%=================================
\section{Data Abstraction}
%\subsection{Game Events Structure}
\begin{frame}
\frametitle{Game Events Structure}
\begin{itemize}
\onslide<1->\item \textbf{Event Category}
\begin{itemize}
\onslide<1->\item There are 8 types of events: catch, miss, dodge, etc.
\onslide<2->\item Categorized by its impact to player's boat into: \textbf{Positive}, \textbf{Neutral}, \textbf{Negative}
\end{itemize}
\onslide<3->\begin{table}[h]
\begin{center}
    \begin{tabular}{| l || l | l | l |}
    \hline
    \textbf{Events} & \textbf{Bonus} & \textbf{Obstacle} & \textbf{Enemy} \\ \hline\hline
    \textbf{Positive} & \textcolor<4>{red}{catch} & - & kill,hit\\ \hline
    \textbf{Neutral}  & miss & dodge & miss\\ \hline
    \textbf{Negative} & - & collision & hurt, collision\\
    \hline
    \end{tabular}
    \label{tblEventType}
\end{center}
\end{table}
\onslide<5->\item \textbf{Screen Speed}: distance between apparition location and event location divided by duration between apparition time and event time
\onslide<6->$$ \upsilon_{scr} = \frac{\theta_{evt}-\theta_{apr}}{\textit{t}_{evt}-\textit{t}_{apr}} $$
\end{itemize}
\end{frame}

%\subsection{Clustering}
\begin{frame}
\frametitle{Clustering - Distance Calculation}

\begin{minipage}[t][0.15\textheight][t]{\textwidth}
\only<2-7>{\[
\textit{s}_1 = \begin{bmatrix}
  \textcolor<7>{red}{10} & 20 & 6\\
  \textcolor<7>{red}{20} & 5 & 18
\end{bmatrix}
\textit{s}_2 = \begin{bmatrix}
  20 & 40 & 10\\ 
  40 & 10 & 30
\end{bmatrix}
\textit{s}_3 = \begin{bmatrix}
  10 & 20 & 5\\
  16 & 4 & 12 
\end{bmatrix}
\]}
\end{minipage}
\begin{minipage}[t][0.85\textheight][t]{\textwidth}
\only<1,2>{\begin{figure}
\includegraphics[scale=0.25]{images/hp_game_divided.png}
\end{figure}}
\begin{itemize}
\onslide<3->\item Let S be a gameplay data set of $n_{ses}$ sessions (Number of rows).
\onslide<4->\item S is an ordered list of sections $s_i, 0 \le i < n_{sec}$ (Number of Matrices)
\onslide<5->\item A section $s_i$ is a sequence of triplets.
\onslide<6->\item The triplet consists of the number of Positive, Neutral and Negative events.
\end{itemize}
\end{minipage}
\end{frame}

\begin{frame}
\frametitle{Clustering - Distance Calculation}
\begin{itemize}
\onslide<1->\item Similarity between two sections is quantified with distance function (distance=0:similar; distance=1: different)
\onslide<2->\item Two types of distance: 
\begin{itemize}
\onslide<3->\item How different both sections in term of event types proportion within each section ( $s_2$ is similar to $s_3$)
\onslide<4->\item How different both sections in term of the evolution of each event type throughout the sessions ( $s_1$ is similar to $s_2$)
\end{itemize}
\only<5,6>{\[
\textit{s}_1 = \begin{bmatrix}
  10 & 20 & 6\\
  20 & 5 & 18
\end{bmatrix}
\textit{s}_2 = \begin{bmatrix}
  \textcolor<6>{red}{20} & \textcolor<6>{red}{40} & \textcolor<6>{red}{10}\\ 
  \textcolor<6>{blue}{40} & \textcolor<6>{blue}{10} & \textcolor<6>{blue}{30}
\end{bmatrix}
\textit{s}_3 = \begin{bmatrix}
  \textcolor<6>{red}{10} & \textcolor<6>{red}{20} & \textcolor<6>{red}{5}\\
  \textcolor<6>{blue}{16} & \textcolor<6>{blue}{4} & \textcolor<6>{blue}{12} 
\end{bmatrix}
\]}

\only<7->{\[
\textit{s}_1 = \begin{bmatrix}
  \textcolor<7>{red}{10} & \textcolor<7>{blue}{20} & \textcolor<7>{green}{6}\\
  \textcolor<7>{red}{20} & \textcolor<7>{blue}{5} & \textcolor<7>{green}{18}
\end{bmatrix}
\textit{s}_2 = \begin{bmatrix}
  \textcolor<7>{red}{20} & \textcolor<7>{blue}{40} & \textcolor<7>{green}{10}\\ 
  \textcolor<7>{red}{40} & \textcolor<7>{blue}{10} & \textcolor<7>{green}{30}
\end{bmatrix}
\textit{s}_3 = \begin{bmatrix}
  10 & 20 & 5\\
  16 & 4 & 12
\end{bmatrix}
\]}

\onslide<8->\item For each pair of consecutive sequences ($\textit{s}_1$,$\textit{s}_2$) of \textit{S}, distance is weighted sum of two distance types, represented as $\textit{f}(\textit{s}_1,\textit{s}_2)$ and $\textit{g}(\textit{s}_1,\textit{s}_2)$.

$$d(\textit{s}_1,\textit{s}_2) = \alpha\textit{f}(\textit{s}_1,\textit{s}_2) + (1 - \alpha)\textit{g}(\textit{s}_1,\textit{s}_2)$$
\end{itemize}


\end{frame}
\begin{frame}
\frametitle{Clustering - Distance Calculation}
\begin{itemize}


\onslide<1->\item $f(\textit{s}_1,\textit{s}_2)$ is a normalized euclidean distance of two triplets.
\onslide<2->\item Distance is the average euclidean distance of each triplets pair. 
$$f(\textit{s}_1,\textit{s}_2) = \frac{\displaystyle\sum_{i=0}^{i < \lvert\textit{s}_1\lvert} \textit{N}ED(\textit{s}_1\lbrack i \rbrack,\textit{s}_2\lbrack i \rbrack)}{ \lvert\textit{s}_1\lvert}$$
\onslide<3->\item Distance is normalized by dividing it with maximum distance $\sqrt{3}$.
$$ \textit{N}ED(\textit{s}_1\lbrack i \rbrack,\textit{s}_2\lbrack i \rbrack) = \frac{\sqrt{\displaystyle\sum_{j \in \{0,1,2\}}(\textit{s}_1'\lbrack i \rbrack\lbrack j \rbrack - \textit{s}_2'\lbrack i \rbrack\lbrack j \rbrack)^2}}{\sqrt{3}}$$
\onslide<4->\item $\textit{s}_1'$ and $\textit{s}_2'$ is normalized value of $\textit{s}_1$ and $\textit{s}_2$ (divided by max value of each session in each section)

\end{itemize}
\end{frame}

\begin{frame}
\frametitle{Clustering - Distance Calculation}
\onslide<1->{\[
\textit{s}_1 = \begin{bmatrix}
  10 & 20 & 6\\
  20 & 5 & 18
\end{bmatrix}
\textit{s}_2 = \begin{bmatrix}
  \textcolor<4,5>{red}{\textbf<1,2,4,5>{20}} & \textcolor<2,4,5>{red}{\textbf<1,2,4,5>{40}} & \textcolor<4,5>{red}{\textbf<1,2,4,5>{10}}\\ 
  \textcolor<3,5>{blue}{\textbf<3,4,5>{40}} & \textcolor<5>{blue}{\textbf<3,4,5>{10}} & \textcolor<5>{blue}{\textbf<3,4,5>{30}}
\end{bmatrix}
\textit{s}_3 = \begin{bmatrix}
  \textcolor<4,5>{red}{\textbf<1,2,4,5>{10}} & \textcolor<2,4,5>{red}{\textbf<1,2,4,5>{20}} & \textcolor<4,5>{red}{\textbf<1,2,4,5>{5}}\\
  \textcolor<3,5>{blue}{\textbf<3,4,5>{16}} & \textcolor<5>{blue}{\textbf<3,4,5>{4}} & \textcolor<5>{blue}{\textbf<3,4,5>{12}} 
\end{bmatrix}
\]}

\onslide<4->{\[
s_1' = \begin{bmatrix}
  \frac{1}{2} & 1 & \frac{6}{20}\\ 
  1 & \frac{1}{4} & \frac{18}{20}
\end{bmatrix}
s_2' = \begin{bmatrix}
  \textcolor<4,5>{red}{\mathbf{\frac{1}{2}}} & \textcolor<4,5>{red}{\textbf<4,5>{1}} & \textcolor<4,5>{red}{\mathbf{\frac{1}{4}}}\\
  \textcolor<5>{blue}{\textbf{1}} & \textcolor<5>{blue}{\mathbf{\frac{1}{4}}} & \textcolor<5>{blue}{\mathbf{\frac{3}{4} }}
\end{bmatrix}
s_3' = \begin{bmatrix}
  \textcolor<4,5>{red}{\mathbf{\frac{1}{2}}} & \textcolor<4,5>{red}{\textbf<4,5>{1}} & \textcolor<4,5>{red}{\mathbf{\frac{1}{4}}}\\
  \textcolor<5>{blue}{\textbf{1}} & \textcolor<5>{blue}{\mathbf{\frac{1}{4}}} & \textcolor<5>{blue}{\mathbf{\frac{3}{4} }}
\end{bmatrix}
\]}
\end{frame}
\begin{frame}
\frametitle{Clustering - Distance Calculation}
\begin{itemize}
\onslide<1->\item $\textit{g}(\textit{s}_1,\textit{s}_2)$ is based on a normalized euclidean distance between the same event type \textit{j} from different section. 
\onslide<2->\item The overall distance of both sections is the average euclidean distance of each event type pair. 
$$\textit{g}(\textit{s}_1,\textit{s}_2) = \frac{\textit{g}_0(\textit{s}_1,\textit{s}_2) + \textit{g}_1(\textit{s}_1,\textit{s}_2) + \textit{g}_2(\textit{s}_1,\textit{s}_2)}{3}$$
\onslide<3->\item For each event type distance, there are three different distance value defined:
\begin{itemize}
\onslide<4->\item if there are no events in both event type pairs, the distance is 0 
\onslide<5->\item if there is at least one event in one section and there are no events in the other section, the distance is 1
\onslide<6->\item if both event type pairs has any events then the euclidean distance is calculated
\end{itemize}

\end{itemize}
\end{frame}
\begin{frame}
\frametitle{Clustering - Distance Calculation}
\onslide<1->for $j  \in \{0,1,2\}$,
\[g_j(\textit{s}_1,\textit{s}_2) = \left\{
  \begin{array}{ll}
    \onslide<2->{0 & \text{ if } \textit{s}_k\lbrack i \rbrack\lbrack j \rbrack=0 \text{ for each } 0 \leqslant j < |\textit{s}_k|\text{ and } \textit{k} \in \{1,2\}}\\
    \onslide<3->{1 & \text{ if } \textit{s}_k\lbrack i \rbrack\lbrack j \rbrack=0 \text{ for each } 0 \leqslant j < |\textit{s}_k| \text{ and } \exists \textit{s}_p\lbrack i \rbrack\lbrack j \rbrack \neq 0,\\
      & k,p \in \{1,2\} \text{, } k \neq p}\\
    \onslide<4->{  & }\\
	\multicolumn{2}{l}{\onslide<4->{\frac{\sqrt{\displaystyle\sum_{i=0}^{\lvert\textit{s}_1\lvert} (\textit{s}_1''\lbrack i \rbrack\lbrack j \rbrack - \textit{s}_2''\lbrack i \rbrack\lbrack j \rbrack)^2}}{\sqrt{\lvert\textit{s}_1\rvert}} \text{ otherwise}} }
  \end{array}
\right.
\]
\begin{itemize}
\onslide<5->\item $s_1''$ and $s_2''$ represent normalized value of $s_1$ and $s_2$ (divided by max value of each event type in each section)

\end{itemize}
\end{frame}

\begin{frame}
\frametitle{Distance Calculation}
\onslide<1->{\[
\textit{s}_1 = \begin{bmatrix}
  \textcolor<7,8,9>{red}{\textbf<1,2,7,8,9>{10}} & \textcolor<4,8,9>{blue}{\textbf<3,4,7,8,9>{20}} & \textcolor<9>{green}{\textbf<5,6,7,8,9>{6}}\\
  \textcolor<2,7,8,9>{red}{\textbf<1,2,7,8,9>{20}} & \textcolor<8,9>{blue}{\textbf<3,4,7,8,9>{5}} & \textcolor<6,9>{green}{\textbf<5,6,7,8,9>{18}}
\end{bmatrix}
\textit{s}_2 = \begin{bmatrix}
  \textcolor<7,8,9>{red}{\textbf<1,2,7,8,9>{20}} & \textcolor<4,8,9>{blue}{\textbf<3,4,7,8,9>{40}} & \textcolor<9>{green}{\textbf<5,6,7,8,9>{10}}\\ 
  \textcolor<2,7,8,9>{red}{\textbf<1,2,7,8,9>{40}} & \textcolor<8,9>{blue}{\textbf<3,4,7,8,9>{10}} & \textcolor<6,9>{green}{\textbf<5,6,7,8,9>{30}}
\end{bmatrix}
\textit{s}_3 = \begin{bmatrix}
  10 & 20 & 5\\
  16 & 4 & 12 
\end{bmatrix}
\]}

\onslide<7->{\[
s_1'' = \begin{bmatrix}
  \textcolor<7,8,9>{red}{\mathbf{\frac{1}{2}}} & \textcolor<8,9>{blue}{\mathbf{1}} & \textcolor<9>{green}{\mathbf{\frac{1}{3}}}\\ 
  \textcolor<7,8,9>{red}{\textbf<7,8,9>{1}} & \textcolor<8,9>{blue}{\mathbf{\frac{1}{4}}} & \textcolor<9>{green}{\mathbf{1}}
\end{bmatrix}
s_2'' = \begin{bmatrix}
  \textcolor<7,8,9>{red}{\mathbf{\frac{1}{2}}} & \textcolor<8,9>{blue}{\mathbf{1}} & \textcolor<9>{green}{\mathbf{\frac{1}{3}}}\\ 
  \textcolor<7,8,9>{red}{\textbf<7,8,9>{1}} & \textcolor<8,9>{blue}{\mathbf{\frac{1}{4}}} & \textcolor<9>{green}{\mathbf{1}}
\end{bmatrix}
s_3'' = \begin{bmatrix}
  \frac{5}{8} & 1 & \frac{5}{12}\\
  1 & \frac{1}{5} & 1 
\end{bmatrix}
\]}
\end{frame}

\begin{frame}
\frametitle{Clustering Algorithm}
\begin{enumerate}
\onslide<1->\item Divide data set into sections the size of x-axis unit
\onslide<2->\item Calculate distance between two consecutive sections
\onslide<3->\item Cluster sections with distance below threshold. This results in a new set of sections
\onslide<4->\item Recalculate distance and cluster
\onslide<5->\item Repeat the process until there is no sections with distance below the threshold
\end{enumerate}
\end{frame}

%=================================
\section{Demo}
\begin{frame}
\frametitle{Demo}
\end{frame}

%=================================
\section{Conclusion}
\begin{frame}
\frametitle{Conclusion}
\begin{itemize}
\onslide<1->\item The proposed visualizations are able to provide information on movement direction, movement and its related events, movement evolution
\onslide<2->\item Case study shows the effectiveness of the interface in answering the questions asked by healthcare professionals
\onslide<3->\item Future Works:
\begin{itemize}
\onslide<4->\item Improve clustering using different distance measure
\onslide<5->\item Study on pathology and its movement to propose game setting based on pathology and its severity
\onslide<6->\item Visualize log data related to skeleton movement
\end{itemize}
\end{itemize}
\end{frame}

\begin{frame}
\frametitle{Question}
\begin{figure}
\includegraphics[scale=1.3]{images/question.png}
\end{figure}
\end{frame}

\begin{frame}
\frametitle{}
\begin{figure}
\includegraphics[scale=0.5]{images/thank_you.png}
\end{figure}
\end{frame}

%==================== Not part of the presentation ===========
%=============================================================
%\subsection{Visualization Requirements}
\begin{frame}
\frametitle{Visualization Requirements}
Tasks related to a session of a particular player:\\
(T1.1) visualize and be able to compare the number of events within the same or among different event type at a given x area (Q1)(Q2). \\
(T1.2) visualize and be able to compare the number of events and its screen speed of the same or among different event type at a given x area (Q1)(Q2)(Q3). \\
(T1.3) select and visualize the number of events for a certain object at a given x area (Q1)(Q2). \\
(T1.4) select and visualize the number of events and its screen speed for a certain object at a given x area (Q1)(Q2)(Q3). 
\end{frame}

\begin{frame}
\frametitle{Visualization Requirements}
Tasks related to the summary of all sessions of a player:\\
(T2.1) visualize, navigate and be able to compare the evolution of number of events throughout all sessions within a certain x area.(Q4)(Q5). \\
(T2.2) select and visualize the number of events of a certain event type in a certain x area throughout all sessions (Q4)(Q5).\\ 
(T2.3) visualize, navigate and be able to compare the distribution of a certain number of events over x area among all sessions(Q4)(Q5). \\
(T2.4) select and visualize the distribution of certain number of events over x area for a certain event type throughout all sessions (Q4)(Q5). \\
(T2.5) extract and visualize similar pattern of number of events evolution throughout all sessions over a certain x area (Q4)(Q5).
\end{frame}

%\section{Related Works}
%\subsection{Visualization of Serious Game Result}
\begin{frame}
\frametitle{Visualization of Serious Game Result}
\begin{figure}
\includegraphics[scale=0.5]{images/rahman_viz.png}
\caption{Line Chart depicting degree of forearm movement over time}
\end{figure}
\end{frame}


%\subsection{Visualization of Time Series Data}

\begin{frame}
\frametitle{Visualization of Time Series Data}
\begin{figure}
\includegraphics[scale=0.5]{images/havre_themeriver.png}
\caption{Theme River}
\end{figure}
\end{frame}

%\subsection{Visualization of Movement Data}

\begin{frame}
\frametitle{Visualization of Movement Data}
\begin{figure}
\includegraphics[scale=0.4]{images/bernard_motionexplorer.png}
\caption{Motion Explorer}
\end{figure}
\end{frame}
%\section{Visual Mappings and Interactive Functionality}
%\subsection{Session Visualization}
\begin{frame}
\frametitle{Session Visualization - Stacked Graph (T1.1)}
\begin{figure}
\includegraphics[scale=0.7]{images/stackedgraph_linear.png}
\caption{Stacked Graph depicting number of events over x axis}
\end{figure}
\end{frame}

\begin{frame}
\frametitle{Session Visualization - Stacked Graph Layout}
\begin{columns}
\column{0.5\textwidth}
\centering
\begin{figure}
\includegraphics[width=5cm,height=2.4cm]{images/stackedgraph_silhouette.png}
\caption{Silhouette}
\end{figure}
\begin{figure}
\includegraphics[width=5cm,height=2.4cm]{images/stackedgraph_positive.png}
\caption{Positive}
\end{figure}

\column{0.5\textwidth}
\centering
\begin{figure}
\includegraphics[width=5cm,height=2.4cm]{images/stackedgraph_neutral_neg.png}
\caption{Neutral-Negative}
\end{figure}
\begin{figure}
\includegraphics[width=5cm,height=2.4cm]{images/stackedgraph_positive_neutral.png}
\caption{Positive-Neutral}
\end{figure}

\end{columns}
\end{frame}

\begin{frame}
\frametitle{Session Visualization - Heatmap (T1.2)}
\begin{figure}
\includegraphics[scale=0.4]{images/heatmap3.png}
\caption{Heatmap depicting number of events and screen speed over x axis}
\end{figure}
\end{frame}

%\subsection{Summary Visualization}
\begin{frame}
\frametitle{Summary Visualization by range of x-area (T2.1)}
\begin{figure}
\includegraphics[scale=0.35]{images/summary_type1.png}
\caption{Summary Visualization by range of x-area: each section has the same x-range}
\end{figure}
\end{frame}
\begin{frame}
\frametitle{Summary Visualization by number of events (T2.3)}
\begin{figure}
\includegraphics[scale=0.35]{images/summary_type2.png}
\caption{Summary Visualization by number of events: each section has the same number of positive and negative events}
\end{figure}
\end{frame}
\begin{frame}
\frametitle{Summary Visualization by clustering (T2.5)}
\begin{figure}
\includegraphics[scale=0.35]{images/summary_clustered.png}
\caption{Summary Visualization by clustering: each section has the similar movement pattern}
\end{figure}
\end{frame}
\begin{frame}
\frametitle{Summary Visualization - Interaction Technique(T2.2,T2.4)}
\begin{figure}
\includegraphics[scale=0.6]{images/interaction_bar.png}
\caption{Interaction bar allows user to choose which event type to show and change input using sliders}
\end{figure}
\end{frame}

\begin{frame}
\frametitle{Summary Visualization - Interaction Technique}
\begin{figure}
\includegraphics[scale=0.4]{images/line_dragging.png}
\caption{By dragging section line or triangle symbol, user can highlight movement pattern }
\end{figure}
\end{frame}
%\subsection{General Interface}
\begin{frame}
\frametitle{General Interface}
\begin{figure}
\includegraphics[scale=0.2]{images/interface_app_compare.png}
\caption{Interface is divided into two areas: navigation area and visualization area}
\end{figure}
\end{frame}

%=================================
%=================================
%\section{Case Study and Demo}
\begin{frame}
\frametitle{Case Study and Demo}
\begin{itemize}
\item Data set is acquired from NaturalPad
\item Data set is of game played by patient with pathology(type of pathology unknown)
\item Case study using HandPoint exercise, consists of 6 sessions
\end{itemize}

\end{frame}

\end{document}
